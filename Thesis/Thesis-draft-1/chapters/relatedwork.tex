\subsubsection{Hot Objects in Encore}
Hot Objects is a concepts to achieve parallelism inside actors, it was implemented in the Encore Programming language but at the time of writing this thesis (spring 2017) the implementation is an old git branch with outdated syntax. Hot Objects is for systems where some actors get a lot more traffic then others. Hot Objects reintroduces locks and allows a single actors to handle more then one massage at the same time. it's important to note that reintroduction of locks only affects the methods inside the Hot object thus the locks are encapsulation inside actors\cite{Hotobjects}. The problems Hot Objects tries to solve is similar to the idea behind the implementing of distributed data types, to run be able to run sequential task in parallel. Even if Hot Objects where up to date with the current Encore Implementation the presence of a the implemented Library would still be interesting because it take another approach to achieve a similar goal.

\subsubsection{Julia's distributed arrays}
Julia is a high-level, dynamic programming language that focus on numeral computing. It provides functionality for distributed parallel execution, numerical accuracy, and an extensive mathematical function library\cite{julia}. The philosophy of Julia is similar to Encores, they both want to provide ways for the programmer to easily achieve parallelism. The library ‘Julia Distributed Arrays’\cite{distarrays} was one of the important inspirations for the this thesis project. The idea is to provide a simple and efficient way to achieve parallelism on large amount of data by divide it up in parts in different processes. The big difference is that Julia’s distributed arrays provide functionality to divide the data between different machine which the implementation in thesis will not cover.

\subsubsection{Java 8 Parallel-streams}
Javas Parallel-streams was first provided in the Java 8 update. The Idea of parallel streams is to make it easier for the program to achieve parallelism. The user specifies how the data should be divided the java run-time then preform partitioning and combines the data \cite{pstreams}. When writing parallel programs in languages with java like threads it can be hard to guarantee that no thread touch each others data, parallels streams ensures the computations to be thread safe. Encore does not have the same problems with thread safety. The implementations of parallel streams probably differ a lot because of how the two different languages creates and handle processes. Parallel streams provide a nice way for the programmer to batch operations together which is something that was explored but not implemented in the work for this thesis.

\subsubsection{Distributed Hash Tables}
Distributed hash tables also called DHT is important for modern distributed systems. Most implementation of DHT does not require a centralized server and treat each node equally. DHT inherits great properties from hash tables and are very efficient even if each node does not have a complete view of the system. DHT is scalable in the aspect that node joining, node leaving and failure are handled \cite{DHT}. The implementing distributed data types in Encore is inspired by classic DHT implementation even if it usage possible differs from a massive DHT network implementations.

\subsubsection{MapReduce}
The style of computing called MapReduce have been implemented in several systems. The term was first introduce by Google with the internal implementation (simply called MapReduce) \cite{mining}. One popular open source implementation is Hadoop created by the Apache Foundation \cite{apgit}. The biggest difference with these MapReduce implemented and the one in Encore is that as of now Encores MapReduce do not support working nodes on different machines. It is hard for Encores MapReduce to compete with the popular implementations because they have for a long time been improved and optimized. It is still a great framework for running big data computations in parallel and fits well with subjects discussed in this thesis.