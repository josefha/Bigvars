\section{Motivation}
Around year 2005 the hardware industry hit a power wall \cite{powerwall}, it was no longer possible to drastically increasing computer performance through decreasing the transistors size or increasing the clock-speed of the CPU-core. To ensure future development multi-core processors became the way to go. Today almost all personal computers have more than one core and to develop programs with strong parallelism support is more important than ever.\\

Parallel programming introduces many problems that the programmer will need to solve. The speed increase gained can be huge even if not all problem benefits from being written in parallel form \cite{paralell-computing}. The Programming Languages Group at Uppsala University are developing a programming language called Encore \cite{encore_paper}, Encore is developed to be scalable to future machines with a few hundred or even thousand processor cores. Encore breaks the norm that programs need to be run sequentially instead of parallel by default. \\

Today arrays and hash tables in Encore are stored in one process resulting in operations on them are executed sequentially. To achieve simple data-parallelism \cite{paralell-computing} one could in Encore dived collections of values in different processes to be able to run tasks on the data concurrently and on different cores.\\

This thesis will examine different types of distributed data types, it will provide abstractions for both the distribution and the preformed parallelism tasks. The distributed data types will store the collection of values distributed among many different processes, operations therefore be execution concurrently and in parallel by default. An easy programmable interface will be provided.\\

It will answers the question, how should this data be divided between processes? So that the programmer don’t need to worry. The Idea is to provide a great tool to fast achieve parallelism to boost performance without much thought. \\

\section{Goal}
Prototype a library implementation of distributed data type written in the Encore Language. It will support more than one distributed data type and should have functionality to make it easier for the programmer to write programs that run concurrently and in parallel. Implement programs that use at least one of these data types in an optimal way. Evaluate and tests the usability and speed, discuss and determine if the implementation is necessary and good enough for future development or if other approaches is more suitable for the Encore Language. 