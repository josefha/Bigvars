\section{Julia's distributed arrays}
Julia is a high-level, dynamic programming language that focus on numeral computing. It provides functionality for distributed parallel execution, numerical accuracy, and an extensive mathematical function library\cite{julia}. The philosophy of Julia is similar to Encores, they both want to provide ways for the programmer to easily achieve parallelism. The library ‘Julia Distributed Arrays’\cite{distarrays} was one of the important inspirations for the Bigvars project, specially the Bigvar arrays. The idea is that a simple and efficient way to achieve parallelism on large amount of data is to divide up on parts in different processes. The big difference is that Julia’s distributed arrays provide functionality to divide the data between different machine which Bigvars do not. 
\section{Java 8 Parallel-streams}
Javas Parallel stream was first provided in the Java 8 update. The Idea of parallel streams is to make it easier for the program to achieve parallelism.The user specifies how the data should be divided the java run-time then preform partitioning and combining to the data.\cite{pstreams}Parallels streams ensures the computations to be thread safe, when writing parallel programs in languages with java threads it can be hard to guarantee that no thread touch each others data. Encore does not have the same problems with thread safety but the idea of parallel streams is still similar to Bigvars except for that the user here is needed to provide how to divide the data. The implementations of parallel streams probably differ a lot because of how the different languages creates processes. Parallel streams also provide a nice way to batch operations together which is something that Bigvars projects explored but not fully implemented. 

\section{Lucas master thesis}

\section{Distributed Hash Tables Implementations}
%% https://gupea.ub.gu.se/bitstream/2077/4580/1/gupea_2077_4580_1.pdf

\section{MapReduce Implementations}
The style of computing called MapReduce have been implemented in several systems. The term was first introduce by Google with the internal implementation (simply called MapReduce) \cite{mining}. One popular open source implementation is Hadoop created by the Apache Foundation \cite{apgit}. The biggest difference with these MapReduce implemented and the one in Encore are that as of now Encores MapReduce do not support working nodes on different machines. It is hard to compete with the popular implementations of MapReduce because they have for a long time been improved and optimized. It is still a great framework for running big data computations in parallel and fits well to the Encore environment. The Encore implementation could always be improved and should take as is has been inspirations from the well known implementations. 