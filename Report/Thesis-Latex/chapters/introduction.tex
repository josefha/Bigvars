Around the year 2005 the hardware industry hit a power wall [1], it was no longer possible to drastically increasing computer performance through decreasing the transistors size or increasing the clock-speed of the CPU-core. To ensure future development multi core CPUs became the way to go. Today almost all personal computers have more than one core and to develop programs with strong parallelism support is more important than ever before.\\

Parallel-programming is not easy and it introduces many new problems that the programmer needs to solve. Speed increase can be huge but not all problem benefits from being written in parallel form [2]. The Programming Languages Group at Uppsala University are developing a programming language called Encore [3], Encore is developed to be scalable to future machines with a few hundred or even thousand CPU-cores [4]. Encore will automatically solve many parallel-problems for the programmer. 
Today arrays in Encore are stored in one process resulting in operations on arrays are executed sequentially. To achieve simple data-parallelism [5] one could in Encore dived collections of values in different processes to then be able to run tasks with the data on different cores. \\

This theses we will examine different types of distributed datatypes, This collection of datatypes is going to be named Bigvars in Encore. Bigvars store a collection of values that are distributed among many processes, different operations on Bigvars could therefore be executed in parallel by default. \\

Bigvars provides answers to the question, how should this data be divided between processes? So the programmer don’t need to worry. Bigvars also have more functionality than normal arrays and hash tables and is thats makes it a great tool to fast achieve parallelism to boost performance without much thought. \\

A implementation of the popular big data framework MapRecude also is exist in the Bigvar environment [] and is yet another feature to make it easy for the programmer to write great programs with parallelism support. 
in the context of the European project Upscale.. 